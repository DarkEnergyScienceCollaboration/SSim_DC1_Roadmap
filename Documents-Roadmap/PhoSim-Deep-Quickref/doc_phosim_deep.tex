\documentclass{article}
\usepackage[letterpaper,margin=1in]{geometry}
\usepackage{graphicx}
\usepackage{xcolor}
\setlength{\parindent}{4em}
\setlength{\parskip}{1em}
\usepackage[
  colorlinks,
  breaklinks,
  pdftitle={Javier S\'{a}nchez - Documentation about PhoSim Deep outputs},
  pdfauthor={Javier S\'{a}nchez},
  unicode
]{hyperref}

\author{Javier S\'{a}nchez \\ francs1@uci.edu \\ University of California Irvine}
\title{Documentation about PhoSim Deep outputs}
\begin{document}
\maketitle
\section{Disclaimer}
This document is intended to be a quick reference for the PhoSim deep outputs. I will try to describe the
different products to the best of my knowledge. Some of these are particular to the way that PhoSim and
the DM stack are run. Any input, clarifications, extensions, etc. are welcome.
\section{Data organization}
The data is contained in the folder \texttt{single\_raft}. This folder is organized as follows:
\begin{itemize}
\item \texttt{image\_repo}: This folder contains the images created by phosim and organized to use \texttt{processEimage.py}, which is a command-line python script that allows us to use the DM stack on phosim-like simulated images. It outputs calibrated exposures and data catalogs. More information here: \url{https://confluence.lsstcorp.org/display/LSWUG/Process+PhoSim+Images} .
\item \texttt{output\_repo}: This folder contains the outputs from the DM stack including: calibrated single exposures, coadds, coadded catalogs, etc. I will cover the particular contents of each folder in section \ref{sec:output_repo}.
\item \texttt{phosim\_images}: This folder contains the images created by phosim (the same as in \texttt{image\_repo}). This folder is particular to this run.
\item \texttt{phosim\_inputs}: This folder contains the input phosim catalogs and it is particular to this run. The catalogs are ASCII files where each line (row) corresponds to a different object. The number of columns varies depending on the nature of the source. Stars have less columns than galaxies. The description of these columns can be found here: \url{https://confluence.lsstcorp.org/display/PHOSIM/Instance+Catalog}. These catalogs are generated from the CatSim database (\url{https://www.lsst.org/scientists/simulations/catsim}) and contain information about position, size, shape and SED of each object, as well as observation conditions: pointing position, airmass... An example python snippet to deal with these files is:
\begin{verbatim}
import pandas as pd
dataframe = pd.read_table('phosim_input_r_0921297_2.5deg.txt', skiprows=20, 
delim_whitespace=True, header=None, names=['object', 'id', 'ra','dec', 'mag_norm',
'sed_name', 'redshift', 'gamma1', 'gamma2', 'kappa', 'delta_ra', 'delta_dec', 
'source_type', 'a', 'b', 'theta', 'n', 'dust_rest_name', 'A_v', 'R_v', 'dust_lab_name'])
star_sel = dataframe['sed_name'].str.contains('star')
gal_sel = dataframe['sed_name'].str.contains('galaxy')
\end{verbatim}
\end{itemize}
\section{Output repo}
\label{sec:output_repo}
In this section I will describe the contents of \texttt{output\_repo}. This folder contains the following subdirectories:
\begin{itemize}
\item \texttt{calexp}: This folder contains different subdirectories with each one of the calibrated exposures (background subtracted + flattened). Each subdirectory contains the resulting science single epoch images for the different sensors (\texttt{S*.fits}) as well as the corresponding background images (\texttt{bkgd*.fits}). This folder is automatically generated but its name can change in different runs.
\item \texttt{config}: This directory contains the scripts used to perform the image processing (coadd, detection/deblending, measurements...). These scripts are available in GitHub. More details and information can be obtained from Jim Chiang, posting questions about them at \url{community.lsst.org} or at slack.
\item \texttt{deepCoadd}: This (generic) folder contains the coadded images. In this case, each fits file correspond to a different sensor: \texttt{0,0.fits; 1,1.fits...}. The folder also contains a ds9 regions file which contains some diffraction spikes as green circles in the file \texttt{1,1.fits} (see Figure \ref{fig:s11_reg}).

\begin{figure}
\centering
\includegraphics[width=0.5\textwidth]{s11}
\caption{Example image showing a small area from \texttt{1,1.fits} and the ds9 regions file contained in the \texttt{deepCoadd} repository.}
\label{fig:s11_reg}
\end{figure}

\item \texttt{deepCoadd-results}: This (generic) folder contains the final catalogs, science images and coadded background images. It contains two subfolders. The first one is called \texttt{merged}: this directory would contain the catalogs generated by multi-band photometry (using \texttt{multiband.py}). In this case, since we only have one band, the catalog contents are the same as in the \texttt{r}  directory. First, it generates a merged catalog of object positions from all input catalogs (\texttt{mergeDet*.fits}). Then, using this catalog as a basis, photometry is performed for the stacked images in each filter, and a new catalog containing the objects found in all filters is generated (\texttt{ref*.fits}). So, for our purposes, the \texttt{ref*.fits} files are the final product. More information about these files can be found here \url{http://hsc.mtk.nao.ac.jp/HSC_Training_tutorial_en/index.html} and here \url{http://hsc.mtk.nao.ac.jp/pipedoc_e/e_usage/index.html}.

The second folder \texttt{r} contains the resulting outputs from the coadding in the r-band. The \texttt{bkgd.fits} files contain the background images, the \texttt{calexp.fits} files contain the calibrated coadd images. The file \texttt{meas*.fits} contains a catalog of objects detected in the coadd image. The file \texttt{det*.fits} contains a list of the same objects. The file \texttt{srcMatch*.fits} contains a match list between the detected objects and the astrometry catalog. More information and details can be found here: \url{http://hsc.mtk.nao.ac.jp/pipedoc_e/e_usage/stack.html#stack}. Remember that since we only have one band you can use \texttt{meas*.fits} as your final catalog as well but, you should take into account that these catalogs might still contain some artifacts.

\item The directories \texttt{*metadata} contain metadata for the different scripts that have been used to process and reduce the data. Some of these folders are particular to this run but most of them are generic to any DM stack run.
\item  The folder \texttt{forced} contains catalogs produced after using \texttt{stack.py} for 1 visit and 1 CCD. I think that this folder is generic. \footnote{Since we are simulating images for individual CCDs these products might look like duplicated data but, in general, that is not the case.}
\item I don't know exactly what the folders \texttt{icExp}, \texttt{icSrc}, \texttt{src}, and \texttt{srcMatch} are meant for but, it looks like they are intermediate products from the different processes. These folders look particular to this run.
\item The folder \texttt{schema} contains information to define the structure of the catalogs. This folder is generic.
\end{itemize}
\section{Notes about \texttt{ref*.fits} columns}
The columns included in these files are the default columns using the baseline DM stack. They include position (RA, Dec), \texttt{psfflux}, aperture fluxes and some SDSS based shape measurements, as well as multiple flags to document the problems that may have arisen during the detection and measurement. The description of these flags is written in the \texttt{TFDOC} fields in the first HDU header. The preferred way to interact with these catalogs is using the butler (see \url{http://www.astro.princeton.edu/~jbosch/hsc/HSC-Pipeline-Outputs.html}). As for now, these files are small and can be ingested with software like \texttt{astropy} but, as the size of these files grow, it is going to be more and more difficult to deal with them.
\end{document}